\section{The ACFE and the CFE Exam}
\label{sect:acfe_and_cfe_exam}


The ACFE, (\url{www.acfe.com}), describes itself on its website as ``the world's largest anti-fraud organization and premier provider of anti-fraud training and education''.  Generally speaking, in order to become a readily employable expert in the field of fraud detection, certification by this organization is required.  The ACFE has well-defined requirements for becoming certified, based on a point system that considers a combination of professional experience and academic credentials.  However, the CFE exam is the credentialing centerpiece for the ACFE, and the details of this exam are described briefly below.

The CFE Exam is a computer-based exam.  The mechanics for preparing for and taking the exam begins with downloading a software package from the \href{http://www.acfe.com/CFE-Exam-Prep-Course-List.aspx}{prep course page of the ACFE website}.  This package includes the exam software, the Fraud Examiners Manual (on which the test is based), and a self-study application consisting of a battery of sample test questions, a complete practice exam, and tools for monitoring progress.  

The CFE exam consists of 4 sections, listed below:

\begin{itemize}
\item Financial Transactions and Fraud Schemes 
\item Law 
\item Investigation
\item Fraud Prevention and Deterrence.  
\end{itemize}

Each section consists of 125 multiple choice and true-false questions.  The candidate is limited to 75 seconds to complete each question and a maximum total allocated time of 2.6 hours to complete each section.  Each CFE Exam section is taken separately.  The timing for each section is subject to the candidate’s discretion.  However, all four sections of the exam must be completed and submitted to the ACFE for grading within a 30 day period.

The goal was to develop a computer system that could pass the CFE exam subject to the constraints outlined above.  This project involved procuring the study package and exam resources, customizing their format for automated processing, and developing and testing the CFE Agent, an extensible system that utilizes various algorithms along with these resources to answer test questions on a given exam.