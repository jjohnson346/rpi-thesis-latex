\section{Introduction}
\label{sect:introduction}

Psychometric AI, or PAI, is grounded on the notion that by making machines that can pass tests, the machine can, in some sense, be called ``intelligent''\cite{bringsjord_schimanski_2004}.  Although it's readily conceded that the ability to pass a particular test (or particular set of tests) is an inadequate basis for defining the full range of what it means to be intelligent, it does offer the very attractive element of concreteness\cite{bringsjord_schimanski_2004} --- a very appealing feature given the struggle one commonly has when attempting to delineate what it means to make machines that can ``think''.  After all, tests offer a number of concrete aspects --- a well-defined start, end, set of goals implied by the questions themselves, and ultimately, a score.  

Evans implicitly began a tradition of test-taking agents by creating ANALOGY in 1968, an agent that could answer geometric analogy questions\cite{evans_1968}.  PERI, an agent created by RPI's RAIR lab in early 2000s, was a robotic agent that could successfully answer Raven's Progressive Matrices questions\cite{raven_1962} as well geometric shape questions similar to those found on the WAIS (Wechsler Adult Intelligence Scale) exam \cite{bringsjord_schimanski_2004}.  Attempts to create such agents even explored the creative realm, such as the one developed to generate stories in a test-based context\cite{bringsjord_schimanski_2004,bringsjord_ferrucci_2000,torrance_1990}.  

The agent described in this paper is designed to take a professional exam, the CFE Exam, that tests knowledge in the fraud detection domain.  This paper will provide some context for the CFE Exam by giving a brief description of its sponsoring organization, the Association of Certified Fraud Examiners (ACFE), and the general features of the test.  Then, it will discuss the approach for building the Agent its performance, as well as directions for future research.






