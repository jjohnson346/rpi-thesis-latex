%%%%%%%%%%%%%%%%%%%%%%%%%%%%%%%%%%%%%%%%%%%%%%%%%%%%%%%%%%%%%%%%%%% 
%                                                                 %
%                            ABSTRACT                             %
%                                                                 %
%%%%%%%%%%%%%%%%%%%%%%%%%%%%%%%%%%%%%%%%%%%%%%%%%%%%%%%%%%%%%%%%%%% 
 
\specialhead{ABSTRACT}
 
This document proposes work which explores an approach for programmatic knowledge base development
and verification using an intelligent agent that takes a multiple choice test using the knowledge base.
A knowledge base is the basis upon which knowledge representation and reasoning systems are built, 
and are thus key to the success and effectiveness of such systems.  Oftentimes, however, knowledge
bases must be developed manually, hand-sewn by knowledge engineers and subject matter experts.
The work proposed here explores programmatic knowledge base development using multiple choice questions about the problem domain, in conjunction with
the traditional methods for entity recognition and
relation extraction.  The purpose of this work is threefold: First, establish an approach for efficient 
knowedge base development.  This will be accomplished by demonstrating this approach on an example domain - fraud detection.  Second, propose and explore a new means of verifying the knowledge base - one which
measures the performance of an agent that takes a multiple choice exam using the knowledge base as a core component, and which provides justifications for its answers.  And third, lay a foundation
for the rigorous science of fraud detection.



