%%%%%%%%%%%%%%%%%%%%%%%%%%%%%%%%%%%%%%%%%%%%%%%%%%%%%%%%%%%%%%%%%%% 
%                                                                 %
%                            CHAPTER FIVE -    Version 3                       %
%                                                                 %
%%%%%%%%%%%%%%%%%%%%%%%%%%%%%%%%%%%%%%%%%%%%%%%%%%%%%%%%%%%%%%%%%%% 
 
\chapter{Version 3 - (Work to be Done) - Toward a Foundation for the Rigorous Science of Fraud Detection}

The notion of formal specifications for fraud and deception are not new, as these topics are explored in Firozabadi \cite{firozabadi1999formal} (fraud), and in Clark \cite{Clark:2010:CIL:2019791} (mendacity).  In this version of the reasoning agent, a formal theory of 
fraud will be developed, expressed in the $\mathcal{DCEC}^\ast$ \cite{ka_sb_scc_seqcalc, mgmmm_ptai_sb} and the agent will detect whether fraud has occurred by
a formal proof-based argument.   The $\mathcal{DCEC}^\ast$ is logico-computational framework for 
knowledge, beliefs, etc., among multiple interacting agents.  This version will also incorporate some form of formal semantics of uncertainty, using any of the formal forms of probability semantics listed above.  In this case, the agent will be tested against manually generated questions for verification.

