%%%%%%%%%%%%%%%%%%%%%%%%%%%%%%%%%%%%%%%%%%%%%%%%%%%%%%%%%%%%%%%%%%% 
%                                                                 %
%                            ABSTRACT_SHORT                             %
%                                                                 %
%%%%%%%%%%%%%%%%%%%%%%%%%%%%%%%%%%%%%%%%%%%%%%%%%%%%%%%%%%%%%%%%%%% 
 
\specialhead{ABSTRACT}
 
This thesis explores various approaches for developing an intelligent agent in a particular domain: fraud detection.  The framework by which we measure our agent is \textit{psychometric artificial intelligence}, or, more commonly, psychometric AI, which focuses on the development of agents that can successfully pass tests.  This approach offers a number of
benefits, including a well-defined domain, a built-in measure for quantifying the efficacy
of the agent in the form of a test score, and a rich environment for deploying various
forms of AI, including machine learning, natural language processing, computer vision,
et cetera.  In this work, we set our sights on one particular test in the fraud-detection domain -- the Certified Fraud Examiners (CFE) exam, administered by the Association of Fraud Examiners 
(ACFE), the governing body overseeing the fraud examiners profession.  The approaches employed in the work presented herein range over a variety of techniques, from extremely shallow text-processing techniques to deep, cognitive, semantic-representation techniques. Version 1 of the agent focuses on shallow text processing techniques that leverage features of the exam and the high-level structure of the Fraud Examiners Manual (FEM) document.  Version 2 employs more sophisticated information retrieval-based approaches wherein the agent breaks up the FEM into more granular documents using the FEM's table of contents and text features.  Version 3 refines the techniques of Version 2 by incorporating machine learning to target the precise paragraphs within the FEM relevant to each question.  Finally, version 4 features an agent with a deep, semantic representation of a subdomain of fraud detection, (doctor shopping), whose knowledge-base consists of assertions expressed in the \textit{deontic cognitive event calculus} $\mathcal{DCEC}^\ast$.  