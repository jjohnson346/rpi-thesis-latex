%%%%%%%%%%%%%%%%%%%%%%%%%%%%%%%%%%%%%%%%%%%%%%%%%%%%%%%%%%%%%%%%%%% 
%                                                                 %
%                            ABSTRACT                             %
%                                                                 %
%%%%%%%%%%%%%%%%%%%%%%%%%%%%%%%%%%%%%%%%%%%%%%%%%%%%%%%%%%%%%%%%%%% 
 
\specialhead{ABSTRACT}
 
This dissertation explores various approaches for developing an intelligent agent in a particular domain: fraud detection.  The framework by which we measure our agent is \textit{psychometric artificial intelligence}, or, more commonly, \textit{psychometric AI} which focuses on the creation of agents that can successfully pass tests.  This approach offers a number of benefits, including a well-defined domain, a built-in measure for quantifying the efficacy of the agent in the form of a test score, and a rich environment for deploying various forms of AI, including machine learning, natural language processing, computer vision, et cetera.  In this work, our attention we'll be centered around natural language processing (NLP).

As part of our commitment to this psychometric approach, we set our sights on one particular test in the fraud-detection domain -- the Certified Fraud Examiners (CFE) exam, administered by the Association of Fraud Examiners (ACFE), the governing body overseeing the fraud examiners profession.  The CFE exam is a multiple-choice exam whose questions are based on the material of the Fraud Examiners Manual (FEM), herein referred to as the CFE Manual.  Programmatic processing of both the CFE Manual and of a training set of CFE exam questions generate the basis of the agent's knowledge and decisions for answering new questions on the test.

The approaches employed in the work presented herein range over a variety of techniques, from extremely shallow text-processing techniques to deep, cognitive, semantic-representation techniques.  Version 1 of the agent focuses on shallow text processing techniques that leverage features of the exam and the high-level structure of the CFE Manual.  Analysis of these shallow algorithms on a training set is then used to apply these algorithms optimally on a test set.  As we'll discuss, even these relatively simplistic techniques generate surprisingly decent scores, although not ones at the level of passing.  Version 2 employs more sophisticated techniques than Version 1, using information retrieval-based approaches to the question-answer task, wherein the agent breaks up the CFE Manual into more granular documents using the CFE Manual's table of contents and text features.  Version 3 incorporates machine learning to target the precise paragraphs within the CFE Manual relevant to each question.  Finally, Version 4 features an agent with a deep, semantic representation of a subdomain of fraud detection, doctor shopping, and whose knowledge-base consists of assertions expressed in the \textit{deontic cognitive event calculus} $\mathcal{DCEC}^\ast$.  The intent in this version is to demonstrate for an example subdomain how formal representation and logico-deductive methods provide transparent justifications and fraud detection. 

This dissertation attempts to advance the field of AI in three ways:  First, introduce and measure the performance of new algorithms in the AI subfield of question-answer (QA).  Second, provide a comparative analysis of these approaches in terms of performance and their ability to provide reasoning justification, and by doing so address an important question in QA -- What are the various marginal costs/benefits associated with state of the art AI approaches in terms of the competing priorities of accuracy, complexity, and provision for reasoning justification.  And third, lay the groundwork for the rigorous study of fraud detection using the formal representation and reasoning approach demonstrated in Version 4, which is not an agent to be measured against exam questions, but one that provides a demonstration of the formal approach to fraud.  It is hoped that by the end of this exploration, the reader will have a solid understanding of the various techniques discussed herein, an appreciation of the benefits of using psychometric AI as the backdrop for intelligent-agent development, and some insights into the relative potential and costs of each of these approaches.